\documentclass{ctexart}
\textheight 23.5cm \textwidth 15.8cm
\topmargin -1.5cm \oddsidemargin 0.3cm \evensidemargin -0.3cm

\usepackage{verbatim}
\usepackage{fancyhdr}
\usepackage{float}
\usepackage{graphicx}
\usepackage{amssymb}
\usepackage{amsmath}
\usepackage{hyperref}
\hypersetup{hidelinks}


\pagestyle{fancy}
\CTEXsetup[format = {\Large\bfseries\it}]{section}

\begin{document}
\section*{内容简介}
	\noindent 利用复化梯形积分公式和复化三点 Gauss 积分公式编写计算积分的通用程序,并计算下列积分:
	\begin{itemize}
		\item $ \displaystyle I_1(f) = \int_0^1 e^{−x^2} dx $
		\item $ \displaystyle I_2(f) = \int_0^4 \dfrac{1}{1 + x^2} dx $
		\item $ \displaystyle I_3(f) = \int_0^{2\pi} \dfrac{1}{2 + \cos(x)} dx $
	\end{itemize}
	取结点 $x_i$,$i = 0, \cdots, N$,$N$ 为 $2^k$,$k = 1, \cdots, 7$ 给出误差表格,其中收敛阶为 
	\begin{equation*}
		\dfrac{\ln(Error_{old} /Error_{now})}{\ln(N_{now} / N_{old})}
	\end{equation*}
	
\section*{工作环境}
	程序所用语言: {\bf python}
	
	软件: {\bf JupyterLab, Mathematica}
	
	使用的包: {\bf mpmath}\\

\section*{输出结果}
	\begin{verbatim}
		Integrate[E^(- x * x), {x, 0.000000, 1.000000}] = 0.7468241328124270
		Composite Trapezoid method:
		I(k = 1) = 0.7313702518285630 	 Error = 1.545388098386e-02 	 Order = 0.0000
		I(k = 2) = 0.7429840978003812 	 Error = 3.840035012046e-03 	 Order = 2.0088
		I(k = 3) = 0.7458656148456952 	 Error = 9.585179667318e-04 	 Order = 2.0022
		I(k = 4) = 0.7465845967882215 	 Error = 2.395360242055e-04 	 Order = 2.0006
		I(k = 5) = 0.7467642546522942 	 Error = 5.987816013281e-05 	 Order = 2.0001
		I(k = 6) = 0.7468091636378279 	 Error = 1.496917459909e-05 	 Order = 2.0000
		I(k = 7) = 0.7468203905416179 	 Error = 3.742270809142e-06 	 Order = 2.0000
		
		Composite Gauss method:
		I(k = 1) = 0.7468240967018682 	 Error = 3.611055884508e-08 	 Order = 0.0000
		I(k = 2) = 0.7468241324102746 	 Error = 4.021524498050e-10 	 Order = 6.4885
		I(k = 3) = 0.7468241328066848 	 Error = 5.742270248690e-12 	 Order = 6.1300
		I(k = 4) = 0.7468241328123394 	 Error = 8.768565026993e-14 	 Order = 6.0331
		I(k = 5) = 0.7468241328124257 	 Error = 1.362201955843e-15 	 Order = 6.0083
		I(k = 6) = 0.7468241328124270 	 Error = 2.125341463362e-17 	 Order = 6.0021
		I(k = 7) = 0.7468241328124270 	 Error = 3.318997105304e-19 	 Order = 6.0008
		
		Integrate[1 / (1 + x * x), {x, 0.000000, 4.000000}] = 1.3258176636680326
		Composite Trapezoid method:
		I(k = 1) = 1.4588235294117646 	 Error = 1.330058657437e-01 	 Order = 0.0000
		I(k = 2) = 1.3294117647058823 	 Error = 3.594101037850e-03 	 Order = 5.2097
		I(k = 3) = 1.3252534024970781 	 Error = 5.642611709544e-04 	 Order = 2.6712
		I(k = 4) = 1.3256735817329137 	 Error = 1.440819351187e-04 	 Order = 1.9695
		I(k = 5) = 1.3257816256818826 	 Error = 3.603798614981e-05 	 Order = 1.9993
		I(k = 6) = 1.3258086530760484 	 Error = 9.010591983910e-06 	 Order = 1.9998
		I(k = 7) = 1.3258154109515374 	 Error = 2.252716495020e-06 	 Order = 2.0000
		
		Composite Gauss method:
		I(k = 1) = 1.3256909037243096 	 Error = 1.267599437228e-04 	 Order = 0.0000
		I(k = 2) = 1.3256917328820794 	 Error = 1.259307859530e-04 	 Order = 0.0095
		I(k = 3) = 1.3258174178690789 	 Error = 2.457989534737e-07 	 Order = 9.0009
		I(k = 4) = 1.3258176636701031 	 Error = 2.070642775051e-12 	 Order = 16.8570
		I(k = 5) = 1.3258176636680783 	 Error = 4.592596126546e-14 	 Order = 5.4946
		I(k = 6) = 1.3258176636680332 	 Error = 7.182505640318e-16 	 Order = 5.9987
		I(k = 7) = 1.3258176636680326 	 Error = 1.122527455403e-17 	 Order = 5.9997
		
		Integrate[1 / (2 + Cos[x]), {x, 0.000000, 6.283185}] = 3.6275987284684357
		Composite Trapezoid method:
		I(k = 1) = 4.1887902047863914 	 Error = 5.611914763180e-01 	 Order = 0.0000
		I(k = 2) = 3.6651914291880923 	 Error = 3.759270071966e-02 	 Order = 3.9000
		I(k = 3) = 3.6277915166453565 	 Error = 1.927881769208e-04 	 Order = 7.6073
		I(k = 4) = 3.6275987335910123 	 Error = 5.122576778448e-09 	 Order = 15.1998
		I(k = 5) = 3.6275987284684357 	 Error = 3.616826829289e-18 	 Order = 30.3995
		I(k = 6) = 3.6275987284684357 	 Error = 1.803043458253e-36 	 Order = 60.7990
		I(k = 7) = 3.6275987284684357 	 Error = 4.480878338110e-73 	 Order = 121.5980
		
		Composite Gauss method:
		I(k = 1) = 3.6337152835897490 	 Error = 6.116555121314e-03 	 Order = 0.0000
		I(k = 2) = 3.6268604008950978 	 Error = 7.383275733380e-04 	 Order = 3.0504
		I(k = 3) = 3.6275944023937576 	 Error = 4.326074677891e-06 	 Order = 7.4151
		I(k = 4) = 3.6275987283534121 	 Error = 1.150233639204e-10 	 Order = 15.1988
		I(k = 5) = 3.6275987284684357 	 Error = 8.121295469872e-20 	 Order = 30.3995
		I(k = 6) = 3.6275987284684357 	 Error = 4.048589927202e-38 	 Order = 60.7990
		I(k = 7) = 3.6275987284684357 	 Error = 1.006145404963e-74 	 Order = 121.5980
	\end{verbatim}
	
\section*{精度检验}
	\begin{table}[H]
		\centering
		\bigskip
		\begin{small}
			\begin{tabular}{|c|cc|cc|}
				\hline
				n & Trapezoid Method Error & Order & Gauss Method Error & Order\\
				\hline
				1 & 1.5454E-02 & -- & 3.6111E-08 & --\\
				2 & 3.8400E-03 & 2.0088 & 4.0215E-10 & 6.4885\\
				3 & 9.5852E-04 & 2.0022 & 5.7423E-12 & 6.1300\\
				4 & 2.3954E-04 & 2.0006 & 8.7686E-14 & 6.0331\\
				5 & 5.9878E-05 & 2.0001 & 1.3622E-15 & 6.0083\\
				6 & 1.4969E-05 & 2.0000 & 2.1253E-17 & 6.0021\\
				7 & 3.7423E-06 & 2.0000 & 3.3190E-19 & 6.0008\\
				\hline
			\end{tabular}
		\end{small}
		\caption{\label{table1.label} $L_\infty$ 范数意义下 $I_1(f)$ 的精度检验} 
	\end{table}
	
	\begin{table}[H]
		\centering
		\bigskip
		\begin{small}
			\begin{tabular}{|c|cc|cc|}
				\hline
				n & Trapezoid Method Error & Order & Gauss Method Error & Order\\
				\hline
				1 & 1.3301E-01 & -- & 1.2676E-04 & --\\
				2 & 3.5941E-03 & 5.2097 & 1.2593E-04 & 0.0095\\
				3 & 5.6426E-04 & 2.6712 & 2.4580E-07 & 9.0009\\
				4 & 1.4408E-04 & 1.9695 & 2.0706E-12 & 16.8570\\
				5 & 3.6038E-05 & 1.9993 & 4.5926E-14 & 5.4946\\
				6 & 9.0106E-06 & 1.9998 & 7.1825E-16 & 5.9987\\
				7 & 2.2527E-06 & 2.0000 & 1.1225E-17 & 5.9997\\
				\hline
			\end{tabular}
		\end{small}
		\caption{\label{table2.label} $L_\infty$ 范数意义下 $I_2(f)$ 的精度检验} 
	\end{table}
	
	\begin{table}[H]
		\centering
		\bigskip
		\begin{small}
			\begin{tabular}{|c|cc|cc|}
				\hline
				n & Trapezoid Method Error & Order & Gauss Method Error & Order\\
				\hline
				1 & 5.6119E-01 & -- & 6.1166E-03 & --\\
				2 & 3.7593E-02 & 3.9000 & 7.3833E-04 & 3.0504\\
				3 & 1.9279E-04 & 7.6073 & 4.3261E-06 & 7.4151\\
				4 & 5.1226E-09 & 15.1998 & 1.1502E-10 & 15.1988\\
				5 & 3.6168E-18 & 30.3995 & 8.1213E-20 & 30.3995\\
				6 & 1.8030E-36 & 60.7990 & 4.0486E-38 & 60.7990\\
				7 & 4.4809E-73 & 121.5980 & 1.0061E-74 & 121.5980\\
				\hline
			\end{tabular}
		\end{small}
		\caption{\label{table3.label} $L_\infty$ 范数意义下 $I_3(f)$ 的精度检验} 
	\end{table}

\section*{误差分析}
	\noindent {\bf 一、复化3点 Gauss 积分公式的推导及误差}
	
	本实验中使用的复化3点 Gauss 积分公式为:
	\begin{equation*}
		\int_a^b f(x)dx = \dfrac{h}{18} \left[5 \sum_{i = 0}^{2^k - 1} f\left(\dfrac{1 - \sqrt{\frac{3}{5}}}{2} h + x_i\right) + 8 \sum_{i = 0}^{2^k - 1} f\left(\dfrac{1}{2} h + x_i\right) + 5 \sum_{i = 0}^{2^k - 1} f\left(\dfrac{1 + \sqrt{\frac{3}{5}}}{2} h + x_i\right)\right]
	\end{equation*}
	
	其中 $h = \dfrac{b - a}{2^k}$,记结点为 $x_0, \cdots, x_{2^k}$。下面对其作简单推导并分析该公式的误差。对子区间 $[x_{i},\,x_{i+1}]$ 即 $[a + ih,\, a + (i+1)h]$,作变量替换
	\begin{equation}
		x = \dfrac{(t + 1)h}{2} + x_i
	\end{equation}
	
	并应用3点Gauss公式,得
	\begin{equation}
	\label{Gauss_Formula}
	\begin{aligned}
		\int_{x_i}^{x_{i+1}} f(x) dx & = \int_{-1}^{1} f(\dfrac{t + 1}{2} h + x_i) \dfrac{h}{2}dt\\
		& = \dfrac{h}{18}\left[5 f\left(\dfrac{1 - \sqrt{\frac{3}{5}}}{2} h + x_i\right) + 8 f\left(\dfrac{1}{2} h + x_i\right) + 5 f\left(\dfrac{1 + \sqrt{\frac{3}{5}}}{2} h + x_i\right)\right]
	\end{aligned}
	\end{equation}
	
	记 $ g(t) = f(\dfrac{t + 1}{2} h + x_i)$,$t_{i0} = -\sqrt{\dfrac{3}{5}}$,$t_{i1} = 0$,$t_{i2} = \sqrt{\dfrac{3}{5}}$,根据 Hermite 插值,存在一个次数最多为5次的多项式 p,使得
	\begin{equation}
		p(t_{ij}) = g(t_{ij}) \qquad p'(t_{ij}) = g'(t_{ij}) \qquad 0 \leqslant j \leqslant 2
	\end{equation}
	
	误差公式为
	\begin{equation}
		g(t) - p(t) = \dfrac{1}{6!} g^{(6)}(\xi_t)\omega^2(t)
	\end{equation}
	
	其中 $\omega(x) = (t - t_{i0})(t - t_{i1})(t - t_{i2})$,$\xi_t \in (t_{i0},\,t_{i2})$。对等号两端积分并应用积分中值定理可得
	\begin{equation}
		\int_{-1}^{1} g(t) dt - \int_{-1}^{1} p(t) dt = \dfrac{g^{(6)}(\xi)}{6!} \int_{-1}^{1} \omega^2(t) dt = \dfrac{8}{175}\dfrac{g^{(6)}(\xi_i)}{6!}
	\end{equation}
	
	因 Gauss 公式对 p 是准确成立的,有
	\begin{equation}
		\int_{-1}^{1} p(t) dt = A_0 p(t_{i0}) + A_1 p(t_{i1}) + A_2 p(t_{i2}) = A_0 f(t_{i0}) + A_1 f(t_{i1}) + A_2 f(t_{i2})
	\end{equation}
	
	那么
	\begin{equation}
		\int_{-1}^{1} g(t) dt = \sum_{j = 0}^{2} A_j g(t_{ij}) + \dfrac{8}{175}\dfrac{g^{(6)}(\xi)}{6!}
	\end{equation}
	
	作积分换元 $t = 2 \dfrac{x - x_i}{h} - 1$,于是又有
	\begin{equation}
	\label{Gauss_Error}
	\begin{aligned}
		\int_{x_i}^{x_{i+1}} f(x) dx & = \dfrac{h}{2} \int_{-1}^{1} g(t) dt\\
		& = \dfrac{h}{2} \sum_{j = 0}^{2} A_j g(t_{ij}) + \dfrac{h}{2} \dfrac{8}{175} \dfrac{h^6}{2^6} \dfrac{f^{(6)}(\zeta_i)}{6!}\\
		& = \dfrac{h}{2} \sum_{j = 0}^{2} A_j g(t_{ij}) + \dfrac{h^7}{175} \dfrac{f^{(6)}(\zeta_i)}{6! \, 2^4}\\
	\end{aligned}
	\end{equation}
	
	
	对 (\ref{Gauss_Formula}) 式与 (\ref{Gauss_Error}) 式的等号两端求和,于是分别得到复化3点 Gauss 公式 (\ref{Composite_Gauss_Formula}) 和其误差项估计式 (\ref{Composite_Gauss_Error}):
	\begin{equation}
	\label{Composite_Gauss_Formula}
	\begin{aligned}
		\int_a^b f(x)dx & = \sum_{i = 0}^{2^k - 1} \int_{x_i}^{x_{i+1}} f(x)dx\\
		& = \dfrac{h}{18} \sum_{i = 0}^{2^k - 1}\left[5 f\left(\dfrac{1 - \sqrt{\frac{3}{5}}}{2} h + x_i\right) + 8 f\left(\dfrac{1}{2} h + x_i\right) + 5 f\left(\dfrac{1 + \sqrt{\frac{3}{5}}}{2} h + x_i\right)\right]\\
		& = \dfrac{h}{18} \left[5 \sum_{i = 0}^{2^k - 1} f\left(\dfrac{1 - \sqrt{\frac{3}{5}}}{2} h + x_i\right) + 8 \sum_{i = 0}^{2^k - 1} f\left(\dfrac{1}{2} h + x_i\right) + 5 \sum_{i = 0}^{2^k - 1} f\left(\dfrac{1 + \sqrt{\frac{3}{5}}}{2} h + x_i\right)\right]
	\end{aligned}
	\end{equation}
	这里尽可能地减少了乘法地运算次数。
	
	结合 $f^{(6)}$ 的连续性,即存在 $\zeta$,使得 $f^{(6)}(\zeta) = \displaystyle \frac{1}{n} \sum_{i = 0}^{n - 1} {f^{(6)}(\zeta_i)}$,$n = 2^k = \dfrac{b - a}{h}$
	\begin{equation}
	\label{Composite_Gauss_Error}
	\begin{aligned}
		\int_a^b f(x)dx - \sum_{i = 0}^{2^k - 1} \dfrac{h}{2} \sum_{j = 0}^{2} A_j g(t_{ij}) & =\sum_{i = 0}^{2^k - 1} \int_{x_i}^{x_{i+1}} f(x) dx - \sum_{i = 0}^{2^k - 1} \dfrac{h}{2} \sum_{j = 0}^{2} A_j g(t_{ij})\\
		& = \sum_{i = 0}^{2^k - 1} \dfrac{h^7}{175} \dfrac{f^{(6)}(\zeta_i)}{6! \, 2^4}\\
		& = \dfrac{h^7}{175 \times 6! \, 2^4} \sum_{i = 0}^{2^k - 1} {f^{(6)}(\zeta_i)}\\
		& = \dfrac{h^6}{175 \times 6! \, 2^4} (b - a) {f^{(6)}(\zeta)}
	\end{aligned}
	\end{equation}
	
	可知误差项为 $O(h^6)$。实验中对于第一、二个积分最终确实得到了 6 阶代数收敛速度,而第三个积分却不符合此规律。\\
	
	\noindent {\bf 二、复化梯形法则的推导及误差}
	
	这部分推导可参考 \href{Report_06.pdf}{Report\_06.pdf} {\bf 误差分析}。复化梯形法则的收敛项是 $O(h^2)$。同上,对于第一、二个积分实际误差达到了预期的 2 阶代数收敛速度。\\
	
	\noindent {\bf 三、精度调整后的额外测试}
	
	由于复化 Gauss 的 6 阶收敛速度非常快,numpy 包的 float64 结构不适应于本次实验,于是采用 mpmath 包,预先设置浮点数精度为 512,以保证实验不会受浮点数精度限制。但即便如此,对于第三个积分,使用复化梯形公式和复化 Gauss 积分公式所得结果的误差均远超出预期的收敛速度。另使用 Mathematica 编程计算,得到了与 python 完全相同的结果。Mathematica 代码见 \href{mma_test.nb}{mma\_test.nb}
	
	使用本实验中的两个算法对一些简单函数,将精度 mpmath.mp.prec 设置为 1024,迭代的 k 上限增加为 18,进行测试\\
	
	$\displaystyle \int_0^1 x^8 dx$、
	$\displaystyle \int_0^1 e^x dx$、
	$\displaystyle \int_0^1 \sin x dx$、
	$\displaystyle \int_0^{0.9} \arcsin x dx$、
	低精度 $\pi$ 下 $\displaystyle \int_0^{2\pi} \dfrac{1}{2 + \cos(x)} dx$ \\
	
	测试结果详见 \href{prec1024.log}{prec1024.log}。通过直接观察,发现了以下事实:
	\begin{itemize}
		\item 使用低精度 $\pi$ 作积分限。此处使用的是 numpy 包内的 numpy.pi,划分次数较少时两方法的计算结果收敛速度均不稳定,且总是高于预期。随着划分次数的增加,最后复化梯形法则误差符合预计的 2 阶代数精度收敛,复化 Gauss 积分公式误差符合预计的 6 阶代数精度收敛。但原来使用 mpmath.pi 计算时收敛阶过大的现象不再复现。原因尚不明确。\\
		
		\item 猜想对于精度较高的 mpmath.pi, 当划分次数足够高时或许能够达到预计的收敛阶。但最终未能得到预计的收敛阶。另一方面,受主机性能限制,程序运行较慢,无法继续增加划分次数和精度来继续测试。\\
		
		\item 对于积分 $\displaystyle \int_0^1 x^8 dx$ 和 $\displaystyle \int_0^1 e^x dx$,在划分次数足够多后,继续增加划分次数会使复化 Gauss 积分公式误差收敛阶降至 2 阶。这里原因同样尚不明确。
	\end{itemize}
	
	以上问题均未能在此次实验中解决。
	
\section*{总结}
	本次实验测试了复化梯形公式和复化三点 Gauss 积分公式通过个人编写代码实现,在计算机上的表现。增进了对数值积分方法的理解。
	
\section*{参考资料}
	\noindent [1] David R. Kincaid \& E. Ward Cheney. {\it Numerical Analysis: Mathematics of Scientific of Computing Third Edition}, Brooks/Cole, 2002.

\end{document}